\documentclass[a4paper]{book}

\usepackage[DIV=13]{typearea} % Benjamins proposal, higher div saves more paper.
% Eliminating empty pages
\let\cleardoublepage\par % Alternatively: \let\cleardoublepage\clearpage

% Improving line layout, f.ex. eliminating overfull boxes, etc.
\usepackage[tracking, kerning, spacing]{microtype}
\microtypecontext{spacing=nonfrench}

\usepackage{verbatim}
\usepackage[hidelinks]{hyperref}




\begin{document}

\tableofcontents

%%%%%%%%%%%%%%%%%%%%%%%%%%%%%%%%%%%%%%%%%%%%%%%%%%%%%%%%%%%%%%%
%%%%%%%%%%%%%%%%%%%%%%%%%%%%%%%%%%%%%%%%%%%%%%%%%%%%%%%%%%%%%%%
%%%
%%%  Installing the necessary software
%%%
%%%%%%%%%%%%%%%%%%%%%%%%%%%%%%%%%%%%%%%%%%%%%%%%%%%%%%%%%%%%%%%
%%%%%%%%%%%%%%%%%%%%%%%%%%%%%%%%%%%%%%%%%%%%%%%%%%%%%%%%%%%%%%%
\chapter{Installing the necessary software}
We will be using lxc with zfs. Install the following
\begin{verbatim}
emerge zfs libvirtd lxc
\end{verbatim}
The kernel needs a lot of configuration. Enable the following flags in the kernel using the menuconfig:
\begin{verbatim}
cd /usr/src/linux
make menuconfig
flags to come
\end{verbatim}

%%%%%%%%%%%%%%%%%%%%%%%%%%%%%%%%%%%%%%%%%%%%%%%%%%%%%%%%%%%%%%%
%%%%%%%%%%%%%%%%%%%%%%%%%%%%%%%%%%%%%%%%%%%%%%%%%%%%%%%%%%%%%%%
%%%
%%%  ZFS
%%%
%%%%%%%%%%%%%%%%%%%%%%%%%%%%%%%%%%%%%%%%%%%%%%%%%%%%%%%%%%%%%%%
%%%%%%%%%%%%%%%%%%%%%%%%%%%%%%%%%%%%%%%%%%%%%%%%%%%%%%%%%%%%%%%
\chapter{ZFS}
More detailed instructions can be found at
\begin{verbatim}
http://wiki.gentoo.org/wiki/ZFS#One_Hard_Drive
\end{verbatim}
After installing zfs you will want it to start at boot. Do
\begin{verbatim}
rc-update add zfs default
\end{verbatim}
Using zfs is very easy. First add a new harddrive to your machine. Initialize it as a new zpool:
\begin{verbatim}
zpool create pool_name /dev/foo
\end{verbatim}
After that you can create volumes. Probably you will want one volume to be your master container from which all other containers are cloned:
\begin{verbatim}
zfs create pool_name/master
\end{verbatim}
Now do everything you need in order to create the master as described in \autoref{creating_master}. When you are done, take a snapshot of your master:
\begin{verbatim}
zfs snapshot pool_name/master@snapshot_name
\end{verbatim}
Creating clones is super easy:
\begin{verbatim}
zfs clone pool_name/master@snapshot_name pool_name/clone1
zfs clone pool_name/master@snapshot_name pool_name/clone2
zfs clone pool_name/master@snapshot_name pool_name/clone3
zfs clone pool_name/master@snapshot_name pool_name/clone4
zfs clone pool_name/master@snapshot_name pool_name/clone5
...
\end{verbatim}
The rest is done in virsh.

%%%%%%%%%%%%%%%%%%%%%%%%%%%%%%%%%%%%%%%%%%%%%%%%%%%%%%%%%%%%%%%
%%%%%%%%%%%%%%%%%%%%%%%%%%%%%%%%%%%%%%%%%%%%%%%%%%%%%%%%%%%%%%%
%%%
%%%  Creating master
%%%
%%%%%%%%%%%%%%%%%%%%%%%%%%%%%%%%%%%%%%%%%%%%%%%%%%%%%%%%%%%%%%%
%%%%%%%%%%%%%%%%%%%%%%%%%%%%%%%%%%%%%%%%%%%%%%%%%%%%%%%%%%%%%%%
\chapter{Creating master}\label{creating_master}

%%%%%%%%%%%%%%%%%%%%%%%%%%%%%%%%%%%%%%%%%%%%%%%%%%%%%%%%%%%%%%%
%%%%%%%%%%%%%%%%%%%%%%%%%%%%%%%%%%%%%%%%%%%%%%%%%%%%%%%%%%%%%%%
%%%
%%%  Using virsh
%%%
%%%%%%%%%%%%%%%%%%%%%%%%%%%%%%%%%%%%%%%%%%%%%%%%%%%%%%%%%%%%%%%
%%%%%%%%%%%%%%%%%%%%%%%%%%%%%%%%%%%%%%%%%%%%%%%%%%%%%%%%%%%%%%%
\chapter{Using virsh}


\end{document}

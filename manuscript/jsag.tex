%
%  jsag
%
%  Created by franzi on 2013-03-24.
%  Copyright (c) 2013 __MyCompanyName__. All rights reserved.
%
\documentclass[twocolumn]{article}

% Use utf-8 encoding for foreign characters
\usepackage[utf8]{inputenc}

% Setup for fullpage use
\usepackage{fullpage, hyperref}

\usepackage{abstract}
% Uncomment some of the following if you use the features
%
% Running Headers and footers
%\usepackage{fancyhdr}

% Multipart figures
%\usepackage{subfigure}

% More symbols
%\usepackage{amsmath}
%\usepackage{amssymb}
%\usepackage{latexsym}

% Surround parts of graphics with box
\usepackage{boxedminipage}

% Package for including code in the document
\usepackage{listings}

% If you want to generate a toc for each chapter (use with book)
\usepackage{minitoc}

% This is now the recommended way for checking for PDFLaTeX:
\usepackage{ifpdf}

%\newif\ifpdf
%\ifx\pdfoutput\undefined
%\pdffalse % we are not running PDFLaTeX
%\else
%\pdfoutput=1 % we are running PDFLaTeX
%\pdftrue
%\fi

\ifpdf
\usepackage[pdftex]{graphicx}
\else
\usepackage{graphicx}
\fi

\def\trym2{{\it TryM2}}
\def\M2{{\it Macaulay2}}

\title{\trym2, a Web Application for \M2}
\author{Lars Kastner\\ Technische Universit\"at Berlin \\{\small kastner\char`\@math.tu-berlin.de} \and
Franziska Hinkelmann\\Google Germany \\{\small franziska.hinkelmann\char`\@gmail.com} \and 
Michael Stillman\\Cornell University \\{\small mike\char`\@math.cornell.edu} \thanks{Stillman has been supported by NSF grants DMS 08-10909 and 10-02210. Hinkelmann has been supported by NSF award 0635561. Kastner by DFG SPP 1489. 
} }


\date{}

\begin{document}



\ifpdf
\DeclareGraphicsExtensions{.pdf, .jpg, .tif}
\else
\DeclareGraphicsExtensions{.eps, .jpg}
\fi


\twocolumn[
    \maketitle
        \begin{onecolabstract}
            \M2 is a software system devoted to supporting research in
            algebraic geometry and commutative algebra, whose creation
            has been funded by the National Science Foundation since
            1992. It is a command-line tool that has no graphical user
            interface.  This manuscript describes a new web
            application, named \trym2, interfacing to \M2. It requires no
            installation and provides a graphical user
            interface, thereby making \M2 more accessible to
            first time users and students.  \trym2
            has all features that the desktop version
            offers.  In addition, it contains tutorials that explain
            different concepts of algebraic geometry such as Gr\"obner
            bases. Users can also create their own tutorials.  \trym2
            has been used in courses at Cornell University,
            Harvard University, Georgia Tech, and Free University of
            Berlin.
        \end{onecolabstract}
]
\saythanks
\section{Introduction}

\M2 is a software system devoted to supporting research in algebraic
geometry and commutative algebra, whose creation and development have
been funded by the National Science Foundation since 1992~\cite{M2}.
This article describes a new web application, named \trym2,
interfacing to \M2.  It is available at~\cite{webM2}.  This web application
has the advantage that users do not need to download or install any
software. A browser and internet connection are enough to run
calculations in \M2. It has the same functionality as the desktop
version, albeit users might have access to less resources than on
their own machine. Users can upload files, load packages, and generate
files such as images.

Our primary motivation for creating this web application was to
provide an easy to use experience for classroom and student use.
Having students download and install software such as \M2 is time
consuming and inevitably there are some situations for which this
process fails or takes significant effort to get running.
\trym2 is designed to
be able to handle many students at once. To keep it as simple as possible,
there is no sign-up or login needed.

\trym2 was used at the Syzygies
meeting in Berlin in May 2013, with about 70 users. It has been used in many courses, including at Cornell
University, Harvard University, Georgia Tech, and Free University of
Berlin. \trym2 is suitable for both beginners and seasoned experts.

We considered off the shelf solutions, such as the Sage
notebook~\cite{sagenotebook}, but we wanted a more
lightweight solution: one that does not require users to create
new accounts at a web site.

\trym2 runs in modern browsers, including Chrome, Firefox, Safari, and Edge, both on mobile devices and desktop
computers.

\trym2 provides several tutorials. Users can create their own
tutorials and share them with other \M2 users.

\begin{figure*}[htb]
    \includegraphics[width=.99\textwidth]{homeWebsite.jpg}
    \caption{A typical view of \trym2. The left hand
        side shows a tutorial giving an introduction to Gr\"obner
        bases. Text in yellow boxes can be clicked and is then executed by
        \M2 on the server. The complete output of the calculations
        is show on the right hand side. Clicking {\it Home, Tutorial, Editor, About}
        changes the left hand view, {\it Reset, Interrupt}
        reset and interrupt the \M2 session on the server, {\it
        Save} provides both the input and the output of the current
        session to the user as a text file, and {\it Upload File} uploads files
        that can then be accessed by \M2.}
\label{fig:home}
\end{figure*}

In Section 2, we describe the basics of using \trym2.
Section 3 details how to create tutorials.
Providing a program such as \M2 which allows users to
access system resources presents a number of challenges to keep users
from naively or mischievously misusing the system.  In Section 4, we
outline technical details of the implementation of \trym2.

\trym2 is open source, and available on GitHub~\cite{github}.  The
\trym2 server can be run on personal laptops and computers,
institutional servers, or in the cloud, e.g. Amazon Web Services.
Details are provided at \cite{github}.  If you would like more
information on running your own \trym2 instance or using \trym2 in a
course, please contact one of us.  The open source backend
architecture is not specific to \M2.  It can be adapted to any
software with a command line interface.  A version for Singular
is already available online (\cite{trySingular}).

\section{Using \trym2}

An important design decision is to keep the interface as simple as
possible.  Therefore there is no login or registration.

After first opening \cite{webM2}, the user is presented with a split
window~\ref{fig:home}.  The right hand side of this window provides a
shell like environment running \M2. The user can type into it and use
the arrow keys for navigating through the command history.
The tab key works for command completion.

On the left hand side of the window, the user can switch between {\it Home}, {\it
  Tutorial}, {\it Editor}, and {\it About}. {\it Home} shows the available
 tutorials. {\it Tutorial} shows the currently selected
tutorial. Tutorials are interactive and contain executable pieces of
\M2 code that are run by clicking on them. {\it Editor} is a
text area in which one can type \M2 commands and evaluate them.

All code executed, either by clicking on interactive parts in a
tutorial or by entering code in the {\it Editor} window, will appear in 
the shell emulator on the right hand side together with the \M2 output.

The {\it Reset} button resets a \M2 session, i.e., restarts 
the \M2 process: it stops any running calculation, deletes all variables, 
unloads all packages, and then reloads the standard packages.

The {\it Interrupt} button stops a running calculation, without
resetting the \M2 session. This can be used to stop a long running
calculation.

\subsection{Features}

The user can run any command in \trym2 that can be run in the desktop
computer version, including commands that refer to external programs
distributed with \M2, such as gfan, bertini, and phcpack.  \M2 comes
with a large number of contributed \M2 packages.  \trym2 can access
all of these packages.  As in the desktop computer version, one loads
such a \M2 package using {\tt needsPackage}, e.g. {\tt needsPackage
  "BoijSoederberg"}.  The user can upload files. For instance, \M2
packages not included in the \M2 distribution can be uploaded and then
loaded into the user's \M2 session using {\tt needsPackage}.  The {\it
  Upload File} button uploads files to the server where they are
available to the user's \M2 session.  This works for all file types,
including \M2 package files.

\begin{figure*}[htb]
    \includegraphics[width=.95\textwidth]{withGraph.jpg}
    \caption{Screenshot after the user
      has generated an image. Some \M2 packages can generate
      graphs. By invoking the command {\tt displayGraph}, the image is
      generated on the server and displayed to the user.}
    \label{fig:graph}
\end{figure*}

Results from a session can be retrieved by using the {\it Save}
button, or by using copy and paste. If \M2 generates
images such as graphs, those are presented to the user, see Figure \ref{fig:graph}.

There is a maximum number of concurrent users.  If there are not this
many users, the \M2 sessions are usually kept alive for several
days. When a user starts a \trym2 session and visits the web app at a
later time, they will continue the previous session. During times of
high demand, inactive sessions may be kept for a shorter period of
time, to make room for new users.  Occasionally the server must be
rebooted, this will unfortunately remove all active sessions.
Resources are generally more restricted than on a personal computer,
since multiple users are sharing these resources.

\section{Tutorials}

You can create your own tutorials for the web version. If you teach a course,
email your tutorial to your students. They can include your tutorial in their web app by clicking
{\it Load Tutorial} on the website.
If you want to share a tutorial with the community, we would
be happy to include them on the website!

You can use the \M2 package {\it DocConverter}.
It converts a file from {\it SimpleDoc} to the tutorial specific HTML. Please see the
{\it DocConverter} package for instructions and examples.


\section{Internal structure}

We choose to implement the server in Node.js
because of Node.js's event-driven, non-blocking I/O model~\cite{nodejs}.
The Node.js server functions as a bridge between \M2 instances and end users.

For every user we start a new \M2 process and
provide them with the underlying Linux system in a separate virtual environment.
This allows advanced users to
interact with the file system or run shell commands by using \M2's {\tt get} 
command:

\begin{verbatim}
i1 : -- get information on the underlying
     -- operating system
     get "!uname -mrs" 
o1 = Linux 3.13.0-49-generic x86_64

i2 : -- write to a file     
     "results.txt" << "Hello!" << close
o2 = results.txt
o2 : File

i3 : -- list all files in the current
     -- directory
     get "!ls"
o3 = results.txt

i4 : -- obtain the file
     get "!open results.txt"
\end{verbatim}

Technically, this is achieved by using Docker containers~\cite{docker}. Docker implements
a high-level API to provide lightweight containers that run processes in isolation.
We start a new Docker container running \M2 for every user. The Node.js
server communicates with Docker containers via {\it ssh}. This allows us to easily
scale the application as demand grows.

To ease the setup process we provide a virtual machine that contains both the Node.js server
and Docker. This virtual machine is configured using {\tt vagrant}, a tool to create and
configure reproducible and portable development environments~\cite{vagrant}.

\section{Conclusion}

By providing \M2 as a web application, we hope to lower the
 entry barrier for new users, therefore increasing the number 
 of users and fostering research in computational algebraic geometry and commutative algebra. 


We are collaborating with several mathematicians to develop more tutorials.
We plan to substantially increase the number of tutorials in the tutorial database over the next semester.


We envision some form of session sharing which allows to use the web app as a tool for collaborative research.


The application was designed with \M2 in mind, but
the entire structure will work for other command line programs such as Singular~\cite{singular},
and GAP~\cite{GAP4}. A similar web application for Singular is work in progress.

\section{Acknowledgments}

We would like to thank Charles Boyd, Dan Grayson, Greg Smith, and Benjamin Lorenz for
fruitful discussions on system security and on user interface design.

\bibliographystyle{plain}
\bibliography{references}
\end{document}
